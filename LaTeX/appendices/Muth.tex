\newcommand{\econtexRoot}{.}
% The \commands below are required to allow sharing of the same base code via Github between TeXLive on a local machine and ShareLaTeX.  This is an ugly solution to the requirement that custom LaTeX packages be accessible, and that ShareLaTeX seems to ignore symbolic links (even if they are relative links to valid locations)
\providecommand{\econtex}{\econtexRoot/texmf-local/tex/latex/econtex}
\providecommand{\econtexSetup}{\econtexRoot/texmf-local/tex/latex/econtexSetup}
\providecommand{\econtexShortcuts}{\econtexRoot/texmf-local/tex/latex/econtexShortcuts}
\providecommand{\econtexBibMake}{\econtexRoot/texmf-local/tex/latex/econtexBibMake}
\providecommand{\econtexBibStyle}{\econtexRoot/texmf-local/bibtex/bst/econtex}
\providecommand{\notes}{\econtexRoot/texmf-local/tex/latex/handout}
\providecommand{\handoutSetup}{\econtexRoot/texmf-local/tex/latex/handoutSetup}
\providecommand{\handoutShortcuts}{\econtexRoot/texmf-local/tex/latex/handoutShortcuts}
\providecommand{\handoutBibMake}{\econtexRoot/texmf-local/tex/latex/handoutBibMake}
\providecommand{\handoutBibStyle}{\econtexRoot/texmf-local/bibtex/bst/handout}

  
 \documentclass{\econtex}\usepackage{\econtexSetup}\usepackage{\econtexShortcuts}../Switches.tex\provideboolean{StandAlone}\setboolean{StandAlone}{true}\setboolean{BigAndWide}{true}   \begin{document}\bibliographystyle{\econtexBibStyle} \setboolean{DocVersion}{false}  \newcommand{\texname}{Muth}\hfill{\tiny \today} \large
  \section{Formulae Derived from \cite{muthOptimal}}\label{appendix:Muth}

  \begin{verbatimwrite}{\econtexRoot/LaTeX/appendices/MuthPre-muthOptimal.texinput}
  \cite{muthOptimal}, pp.\ 303--304,  shows that the signal-extracted estimate of permanent income is
  \begin{eqnarray}
  \perc{\PLev}_{t} & = & v_{1}Y_{t}+v_{2}Y_{t-1}+v_{3}Y_{t-2}+...
  \end{eqnarray}
  for a sequence of $v$'s given by
  \begin{eqnarray}
    v_{k} & = & (1-\lambda_{1})\lambda_{1}^{k-1}
  \end{eqnarray}
  for $k = 1,2,3,...$.  So:
  \begin{eqnarray}
    \perc{\PLev}_{t} & = & (1-\lambda_{1})(\phantom{\YLev_{t+1}+\lambda_{1}^{0}}\YLev_{t}+\lambda_{1} \YLev_{t-1} + \lambda_{1}^{2} \YLev_{t-2} ...)
\\    \perc{\PLev}_{t+1} & = & (1-\lambda_{1})(\YLev_{t+1}+\lambda_{1} \YLev_{t} + \lambda_{1}^{2} \YLev_{t-1}+ \lambda_{1}^{3} \YLev_{t-2} ...)
\\     & = & (1-\lambda_{1})\phantom{(}\YLev_{t+1}+\lambda_{1} \underbrace{(1-\lambda_{1})(\YLev_{t} + \lambda_{1}^{2} \YLev_{t-1}+ \lambda_{1}^{3} \YLev_{t-2} ...)}_{\perc{\PLev}_{t}}
\\    & = & (1-\lambda_{1})\YLev_{t+1}+\lambda_{1} \perc{\PLev}_{t}
  \end{eqnarray}

This compares with \eqref{eq:PischkeP} in the main text
  \begin{eqnarray}
    \perc{\PLev}_{t+1} & = & \Pi \YLev_{t+1} + (1-\Pi) \perc{\PLev}_{t}
  \end{eqnarray}
so the relationship between our $\Pi$ and Muth's $\lambda_{1}$ is:
  \begin{eqnarray}
    \lambda_{1} & = & 1-\Pi
  \end{eqnarray}

  Defining the signal-to-noise ratio $\varphi = \sigma_{\pmb{\psi}}/\sigma_{\pmb{\theta}}$, starting with equation (3.10) in \cite{muthOptimal} we have
  \begin{eqnarray}
\notag    \lambda_{1} & = & 1 + (1/2) \varphi^{2} - \varphi \sqrt{1+\varphi^{2}/4}
\notag \\ (1-\Pi) & = & 1 + (1/2) \varphi^{2} - \varphi \sqrt{1+\varphi^{2}/4}
                        \notag\\ -\Pi & = & (1/2) \varphi^{2} - \varphi \sqrt{1+\varphi^{2}/4}
  \end{eqnarray}
  yielding equation \eqref{eq:muthOptimal} in the main text.
\end{verbatimwrite}
\input{\econtexRoot/LaTeX/appendices/MuthPre-muthOptimal.texinput}
  \begin{verbatimwrite}{\eq/muthOptimal.tex}
  \begin{eqnarray}
\Pi & = & \varphi \sqrt{1+\varphi^{2}/4} - (1/2) \varphi^{2} \label{eq:muthOptimal}
  \end{eqnarray}
\end{verbatimwrite}
  \begin{eqnarray}
\Pi & = & \varphi \sqrt{1+\varphi^{2}/4} - (1/2) \varphi^{2}
  \end{eqnarray}


\bibliography{economics,\texname,\texname-Add}

\end{document}
