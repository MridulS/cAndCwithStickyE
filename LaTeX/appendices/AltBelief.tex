In the model presented in the main text, households with sticky expectations use the same consumption function as households who frictionlessly observe macroeconomic information in all periods.  They treat their \textit{perceptions} of macroeconomic states as if they were the true values, and do not account for their inattention when optimizing.  In this appendix, we present an alternate specification in which households with sticky expectations partially account for their inattention by optimizing as if the flow of macroeconomic information they will receive is the true aggregate shock process.  Simulated results analogous to Table~\ref{tPESOEsim} in the main text are presented below in Table~\ref{tPESOEsimAlt}.

Sticky expectations households do not update their macroeconomic information a $1-\Pi$ fraction of the time. In these periods, they perceive that there was no permanent aggregate shock $\Psi_t$ and no innovation to the aggregate growth rate $\PtyGro_t$.  When they do update, they learn of the accumulation of permanent aggregate shocks since their last update (compounded with deviations from the last observed aggregate growth rate), as well as the new growth rate.  In the ``alternate beliefs'' specification, households solve for their optimal consumption rule by treating their \textit{perceived flow} of macroeconomic information as the true aggregate process.  In this way, they \textit{partially} account for their inattention by recognizing that the macroeconomic news they will perceive is leptokurtic relative to frictionless households. 

The perceived aggregate shock process on which sticky households optimize is a linear combination of the shocks they perceive in non-updating periods (with weight $1-\Pi$) and the shocks they perceive when they do update (with weight $\Pi$).  In periods in which they do and don't update, households treat the distribution of aggregate shocks as respectively:
\begin{equation*}
	\Theta_t^{\Pi} \sim \mathcal{N}(-\sigma_\Theta^2/2,\sigma_\Theta^2), \qquad \Psi_t^{\Pi} \sim \mathcal{N}(-\sigma_\Psi^2/(2\Pi),\sigma_\Psi^2/\Pi), \qquad \Xi^{\Pi} \sim \Xi^{\lfloor 1/\Pi \rceil}.
\end{equation*}
\begin{equation*}
	\Theta_t^{\bcancel{\Pi}} \sim \mathcal{N}(-(\sigma_\Theta^2 + \sigma_\Psi^2/\Pi)/2,\sigma_\Theta^2 + \sigma_\Psi^2/\Pi), \qquad \Psi_t^{\bcancel{\Pi}} = 1, \qquad \Xi^{\bcancel{\Pi}} \sim I.
\end{equation*}
Here, $\Xi$ represents the transition matrix among discrete Markov states for $\PtyGro_t$ in the true aggregate shock process.  Under sticky expectations, households optimize under the assumption that in the $\Pi$ fraction of periods in which $\PtyGro_t$ is observed, the true transition process has transpired an average of $\lfloor 1/\Pi \rceil$ times since the last update (four, under our calibration); they anticipate no Markov dynamics in the periods when they do not update (identity matrix $I$).  Likewise, aggregate permanent shocks are interpreted to be degenerate in non-updating periods, but to make up for the fact that updating periods are one quarter as common, when an update occurs its variance is four times as large as in the baseline model.

In non-updating periods, households interpret all deviations from expected $\PLev_t$ as transitory aggregate shocks, so their perceived variance of $\Theta_t$ includes both transitory aggregate variance and a geometric series of permanent aggregate variance, decaying at rate $(1-\Pi)$:
\begin{equation*}
\sigma_\Theta^2 + \sigma_\Psi^2 + (1-\Pi) \sigma_\Psi^2 + (1-\Pi)^2 \sigma_\Psi^2 + \cdots = \sigma_\Theta^2 + \sigma_\Psi^2/\Pi.
\end{equation*}

This alternate belief specification does not have sticky expectations households fully and correctly adjust for their inattention.  They do not track the \textit{number} of periods since their last macroeconomic update, instead treating all non-updating periods alike from the perspective of perceived transitory shocks. Households act according to the same consumption function whether or not they just updated; the more sophisticated shock structure is used only to better approximate the perceived arrival of macroeconomic news when solving the problem.  Moreover, households do not account for the positive covariance between accumulated permanent aggregate shocks and the innovation to $\PtyGro_t$ in periods when they \textit{do} update.  Incorporating these calculations would be extremely computationally burdensome, while changing the optimal consumption policy by very little.  To the extent that our model represents an abstraction from households choosing the frequency of updating to balance the marginal cost and benefit of obtaining macroeconomic news (see section~\ref{sec:uCost}), it seems unlikely that agents would then adopt a vastly more complicated view of the world to offset the mild consequences of their inattention.

The key result is that households' optimal consumption function barely changes from baseline when the alternate beliefs are introduced: across states actually attained during simulation, normalized consumption differs by no more than 0.2 percent, and the difference is less than 0.02 percent in the vast majority of states.  More importantly, the macroeconomic dynamics generated by sticky expectations households' collective behavior is nearly identical between the bottom panels of Table~\ref{tPESOEsimAlt} below and Table~\ref{tPESOEsim} in the main text.\footnote{The top panels are literally identical, as they report the same model.}  This experiment represents a more general proposition that our main results should be robust to the details of the precise specification of households' understanding of their inattention, so long as the key feature remains that agents' idiosyncratic errors are \textit{systematically correlated} due to the lag in information.


\input\econtexRoot/Tables/SOEmrkvSimRegAlt.tex
