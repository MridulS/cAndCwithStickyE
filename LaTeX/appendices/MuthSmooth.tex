To see how the Muth--Lucas--Pischke model can generate smoothness, note that in the Muth framework, agents update their estimate of permanent income according to an equation of the form:\footnote{$\hat{P}_t$ is used to denote that households do an optimal signal-extraction (as opposed to having sticky expectations resulting in $\perc{P}_t$).}
\begin{eqnarray*}
	\label{eq:PischkeP}
	\hat{\PLev}_{t+1} & = & \Pi \mathbf{Y}_{t+1}+(1-\Pi) \hat{\PLev}_{t}.
\end{eqnarray*}

We can now consider the dynamics of aggregate consumption in response to the arrival of an aggregate shock that (unbeknownst to the consumer) is permanent.  The consumer spends $\Pi$ of the shock in the first period, leaving $(1-\Pi)$ unspent because that reflects the average transitory component of an undifferentiated shock.  However, since the shock really was permanent, income next period does not fall back as the consumer guessed it would on the basis of the mistaken belief that $(1-\Pi)$ of the shock was transitory.  The next-period consumer treats this surprise as a positive shock relative to expected income, and spends the same proportion $\Pi$ out of the perceived new shock.  These dynamics continue indefinitely, but with each successive perceived shock (and therefore each consumption increment) being smaller than the last by the proportion $(1-\Pi)$.  Thus, after a true permanent shock received in period $t$, the full-information prediction of the expected dynamics of future consumption changes would be $\Delta \mathbf{C}_{t+n+1} = (1-\Pi)  \Delta \mathbf{C}_{t+n} + \epsilon_{t+n}$.\footnote{The reciprocal logic would apply in the case of a shock that was known by the econometrician to be perfectly transitory, generating the same serial correlation in predictable consumption growth as in the case of the known-to-be-permanent shock.  The only circumstance under which this serial correlation does {\it not} arise is when the econometrician has exactly the same beliefs as the consumer about the breakdown of the shock between transitory components.  More precisely, it is still the case that the serial correlation coefficient on the predictable component of consumption growth is $(1-\Pi)$.  But that predictable component itself is now zero, and $(1-\Pi)\times0=0$.}

At first blush, this predictability in consumption growth would appear to be a violation of \cite{hallRandomWalk}'s proof that, for consumers who make rational estimates of their permanent income, consumption must be a random walk.  The reconciliation is that what Hall proves is that consumption must be a random walk {\it with respect to the knowledge the consumer has}.  The random walk proposition remains true for consumers whose knowledge base contains only the perceived level of aggregate income.  Our thought experiment was to ask how much predictability would be found by an econometrician {\it who knows more than the consumer} about the level of aggregate permanent income.

The in-principle reconciliation of econometric evidence of predictability/excess smoothness in consumption growth, and the random walk proposition, is therefore that the econometricians who are making their forecasts of aggregate consumption growth use additional variables (beyond the lagged history of aggregate income itself), and that those variables have useful predictive power.\footnote{This is logically identical to Pischke's analysis of the case where the macroeconometrician knows that aggregate shocks are permanent, but the microeconomic consumers do not perceive those aggregate permanent shocks.}
